\documentclass[UTF8, 12pt]{ctexart}
\usepackage{amsmath}
\usepackage{graphicx}

% 取消段首缩进
\setlength{\parindent}{0pt}

\begin{document}

1.
位置:
\[
    \begin{aligned}
        &\text{B相对于\{A\}为} \; \begin{bmatrix} -100 & 200 & 120 \end{bmatrix}^\top,\\
        &\text{C相对于\{A\}为} \; \begin{bmatrix} 150 & 220 & -120 \end{bmatrix}^\top
    \end{aligned}
\]

姿态:
\[
    \begin{aligned}
        {}^A R_B &= 
            \begin{bmatrix}
            0 & -1 & 0 \\
            0 & 0 & 1 \\
            -1 & 0 & 0
            \end{bmatrix} \\
        {}^A R_C &=
            \begin{bmatrix}
            0 & 1 & 0 \\
            0 & 0 & 1 \\
            1 & 0 & 0
            \end{bmatrix}
    \end{aligned}
\]

方向余弦矩阵为对应旋转变换矩阵转置
\[
\begin{aligned}
    {}^A C_B &= 
        \begin{bmatrix}
        0 & 0 & -1 \\
        -1 & 0 & 0 \\
        0 & 1 & 0
        \end{bmatrix} \\
    {}^A C_C &=
        \begin{bmatrix}
        0 & 0 & 1 \\
        1 & 0 & 0 \\
        0 & 1 & 0
        \end{bmatrix}
\end{aligned}
\]
\newpage


2.
\[
\begin{aligned}
    {}^a R_b &= R_z(-10^\circ) R_x(25^\circ) R_y(30^\circ) \\
    {}^b R_c &= R_z(5^\circ) R_x(15^\circ) R_y(20^\circ) \\
    {}^a R_c &= {}^a R_b \; {}^b R_c \\
            &= R_z(-10^\circ) R_x(25^\circ) R_y(30^\circ) R_z(5^\circ) R_x(15^\circ) R_y(20^\circ) \\
            &= \begin{bmatrix}
                    \cos(-10^\circ) & -\sin(-10^\circ) & 0 \\
                    \sin(-10^\circ) &  \cos(-10^\circ) & 0 \\
                    0 & 0 & 1
                \end{bmatrix}
                \begin{bmatrix}
                    1 & 0 & 0 \\
                    0 & \cos(25^\circ) & -\sin(25^\circ) \\
                    0 & \sin(25^\circ) & \cos(25^\circ)
                \end{bmatrix}
                \begin{bmatrix}
                    \cos(30^\circ) & 0 & \sin(30^\circ) \\
                    0 & 1 & 0 \\
                    -\sin(30^\circ) & 0 & \cos(30^\circ)
                \end{bmatrix} \\[6pt]
                &\phantom{=} \;\;
                \begin{bmatrix}
                    \cos(5^\circ) & -\sin(5^\circ) & 0 \\
                    \sin(5^\circ) &  \cos(5^\circ) & 0 \\
                    0 & 0 & 1
                \end{bmatrix}
                \begin{bmatrix}
                    1 & 0 & 0 \\
                    0 & \cos(15^\circ) & -\sin(15^\circ) \\
                    0 & \sin(15^\circ) & \cos(15^\circ)
                \end{bmatrix}
                \begin{bmatrix}
                    \cos(20^\circ) & 0 & \sin(20^\circ) \\
                    0 & 1 & 0 \\
                    -\sin(20^\circ) & 0 & \cos(20^\circ)
                \end{bmatrix} \\
            &= \begin{bmatrix}
                    0.7110 & 0.1875 & 0.6778 \\
                    0.3531 & 0.7382 & -0.5747 \\
                    -0.6081 & 0.6480 & 0.4586
                \end{bmatrix}
\end{aligned}
\]
\vspace{1em}

对应ZXY欧拉角:
\[
\begin{aligned}
    R &= R_z(\alpha) R_x(\beta) R_y(\gamma) \\
        &= \begin{bmatrix}
                \cos\alpha \cos\gamma - \sin\alpha \sin\beta \sin\gamma & -\sin\alpha \cos\beta & \cos\alpha \sin\gamma + \sin\alpha \sin\beta \cos\gamma \\
                \sin\alpha \cos\gamma + \cos\alpha \sin\beta \sin\gamma & \cos\alpha \cos\beta & \sin\alpha \sin\gamma - \cos\alpha \sin\beta \cos\gamma \\
                -\cos\beta \sin\gamma & \sin\beta & \cos\beta \cos\gamma
            \end{bmatrix} \\
\end{aligned}
\]

解得:
\[
\begin{aligned}
    &\beta = \arcsin(0.6480) = 40^\circ \; \text{或} \; \beta = 180^\circ - 40^\circ = 140^\circ \\
    &\alpha = \arctan2(-0.1875, 0.7382) = -14^\circ \; \text{或} \; arctan2(0.1875, -0.7382) = 166^\circ \\
    &\gamma = \arctan2(0.6081, 0.4586) = 53^\circ \; \text{或} \; \arctan2(-0.6081, -0.4586) = -127^\circ
\end{aligned}
\]
\vspace{1em}

对应XYX欧拉角:
\[
\begin{aligned}
    R &= R_x(\alpha) R_y(\beta) R_x(\gamma) \\
        &=  \begin{bmatrix}
                \cos\beta & \sin\beta \sin\gamma & \sin\beta \cos\gamma \\
                \sin\alpha \sin\beta & \cos\alpha \cos\gamma - \sin\alpha \cos\beta \sin\gamma & -\cos\alpha \sin\gamma - \sin\alpha \cos\beta \cos\gamma \\
                -\cos\alpha \sin\beta & \sin\alpha \cos\gamma + \cos\alpha \cos\beta \sin\gamma & -\sin\alpha \sin\gamma + \cos\alpha \cos\beta \cos\gamma
            \end{bmatrix} \\
\end{aligned}
\]

解得:
\[
\begin{aligned}
    &\beta = \pm \arccos(0.7110) = 45^\circ \\
    &\alpha = \arctan2(0.3531, 0.6081) = 30^\circ \; \text{或} \; \arctan2(-0.3531, -0.6081) = -150^\circ \\
    &\gamma = \arctan2(0.1875, 0.6778) = 15^\circ \; \text{或} \; \arctan2(-0.1875, -0.6778) = -165^\circ
\end{aligned}
\]
\vspace{1em}

即\{c\}相对于\{a\}的ZXY欧拉角为
\[
\begin{bmatrix}
    -14^\circ \\ 40^\circ \\ 53^\circ
\end{bmatrix} \;
\begin{bmatrix}
    166^\circ \\ 140^\circ \\ -127^\circ
\end{bmatrix}
\]
XYX欧拉角为
\[
\begin{bmatrix}
    30^\circ \\ 45^\circ \\ 15^\circ
\end{bmatrix} \;
\begin{bmatrix}
    -150^\circ \\ -45^\circ \\ -165^\circ
\end{bmatrix}
\]

\newpage

3.
(1)
\[
\begin{aligned}
    {}^a R_b &= R_y(20^\circ) R_x(-30^\circ) R_y(40^\circ) \\
            &=
            \begin{bmatrix}
                \cos(20^\circ) & 0 & \sin(20^\circ) \\
                0 & 1 & 0 \\
                -\sin(20^\circ) & 0 & \cos(20^\circ)
            \end{bmatrix}
            \begin{bmatrix}
                1 & 0 & 0 \\
                0 & \cos(-30^\circ) & -\sin(-30^\circ) \\
                0 & \sin(-30^\circ) & \cos(-30^\circ)
            \end{bmatrix}
            \begin{bmatrix}
                \cos(40^\circ) & 0 & \sin(40^\circ) \\
                0 & 1 & 0 \\
                -\sin(40^\circ) & 0 & \cos(40^\circ)
            \end{bmatrix} \\
            &=
            \begin{bmatrix}
                0.5295 & -0.1710 &  0.8309 \\
                -0.3214 &  0.8660 &  0.3830 \\
                -0.7851 & -0.4698 &  0.4036
            \end{bmatrix} \\
    {}^a T_b &= 
        \begin{bmatrix}
            0.5295 & -0.1710 &  0.8309 & -100 \\
            -0.3214 &  0.8660 &  0.3830 & 400 \\
            -0.7851 & -0.4698 &  0.4036 & 150 \\
            0 & 0 & 0 & 1
        \end{bmatrix}
\end{aligned}
\]
\vspace{5em}


(2)
\[
\begin{aligned}
    {}^b \overline{p}_b &=  \begin{bmatrix}
                    -20 \\ 30 \\ -30 \\ 1
                \end{bmatrix} \\
    {}^a \overline{p}_a &= {}^a T_b \; {}^b \overline{p}_b \\
            &=
            \begin{bmatrix}
                0.5295 & -0.1710 &  0.8309 & -100 \\
                -0.3214 &  0.8660 &  0.3830 & 400 \\
                -0.7851 & -0.4698 &  0.4036 & 150 \\
                0 & 0 & 0 & 1
            \end{bmatrix}
            \begin{bmatrix}
                -20 \\ 30 \\ -30 \\ 1
            \end{bmatrix} \\
            &=
            \begin{bmatrix}
                -140.65 \\ 420.92 \\ 139.50 \\ 1
            \end{bmatrix}
\end{aligned}
\]
\vspace{5em}


(3)
\[
    \omega = \begin{bmatrix}
                0 \\ \dot{\alpha} \\ 0
            \end{bmatrix} 
            + R_y(20^\circ) \begin{bmatrix}
                \dot{\beta} \\ 0 \\ 0
            \end{bmatrix}
            + R_y(20^\circ) R_x(-30^\circ) \begin{bmatrix}
                0 \\ \dot{\gamma} \\ 0
            \end{bmatrix} \\
\]
其中
\[
\begin{aligned}
    R_y(20^\circ) &= 
        \begin{bmatrix}
            \cos(20^\circ) & 0 & \sin(20^\circ) \\
            0 & 1 & 0 \\
            -\sin(20^\circ) & 0 & \cos(20^\circ)
        \end{bmatrix} \\
        &=
        \begin{bmatrix}
            0.9397 & 0 & 0.3420 \\
            0 & 1 & 0 \\
            -0.3420 & 0 & 0.9397
        \end{bmatrix} \\
\end{aligned} 
\]
\[
\begin{aligned}
    R_y(20^\circ) R_x(-30^\circ) &=
        \begin{bmatrix}
            \cos(20^\circ) & 0 & \sin(20^\circ) \\
            0 & 1 & 0 \\
            -\sin(20^\circ) & 0 & \cos(20^\circ)
        \end{bmatrix}
        \begin{bmatrix}
            1 & 0 & 0 \\
            0 & \cos(-30^\circ) & -\sin(-30^\circ) \\
            0 & \sin(-30^\circ) & \cos(-30^\circ)
        \end{bmatrix} \\
        &=
        \begin{bmatrix}
            0.9397 & -0.1710 & 0.2962 \\
            0 & 0.8660 & 0.5000 \\
            -0.3420 & -0.4698 & 0.8138
        \end{bmatrix} \\
\end{aligned}
\]

代入得
\[
\begin{aligned}
    \omega &=
        \begin{bmatrix}
            0 \\ \dot{\alpha} \\ 0
        \end{bmatrix}
        +
        \begin{bmatrix}
            0.9397 & 0 & 0.3420 \\
            0 & 1 & 0 \\
            -0.3420 & 0 & 0.9397
        \end{bmatrix}
        \begin{bmatrix}
            \dot{\beta} \\ 0 \\ 0
        \end{bmatrix}
        +
        \begin{bmatrix}
            0.9397 & -0.1710 & 0.2962 \\
            0 & 0.8660 & 0.5000 \\
            -0.3420 & -0.4698 & 0.8138
        \end{bmatrix}
        \begin{bmatrix}
            0 \\ \dot{\gamma} \\ 0
        \end{bmatrix} \\
        &= 
        \begin{bmatrix}
            0.9397\dot{\beta} - 0.1710\dot{\gamma} \\
            \dot{\alpha} + 0.8660\dot{\gamma} \\
            -0.3420\dot{\beta} - 0.4698\dot{\gamma}
        \end{bmatrix}
        = \begin{bmatrix}
            5 \\ 10 \\ -8
        \end{bmatrix}
\end{aligned}
\]

解得
\[
\begin{cases}
    \dot{\alpha} = -0.0597 \\
    \dot{\beta} = 7.4347 \\
    \dot{\gamma} = 11.6163
\end{cases}
\]

即
\[
\dot{\Psi} = \begin{bmatrix}
    -0.0597 \\ 7.4347 \\ 11.6163
\end{bmatrix}
\]
\vspace{5em}


(4)
\(\beta = 0^\circ \) 或 \(\beta = 180^\circ\)时,YXY欧拉角姿态奇异

\end{document}