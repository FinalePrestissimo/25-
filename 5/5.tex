\documentclass[UTF8, 12pt]{ctexart}
\usepackage{amsmath}
\usepackage{graphicx}
\usepackage{physics}

% 取消段首缩进
\setlength{\parindent}{0pt}

\begin{document}

1.
(1)
由于
\[
{}^0 J = \begin{bmatrix}
            {}^0 R_n & 0 \\
            0 & {}^0 R_n
        \end{bmatrix} {}^n J
\]
其中\({}^0 R_n\)为正交矩阵,故
\[
\begin{aligned}
    {}^n J &= \begin{bmatrix}
            {}^0 R_n & 0 \\
            0 & {}^0 R_n
            \end{bmatrix}^{-1} 
            {}^0 J \\
    &= \begin{bmatrix}
            {}^0 R_n^{-1} & 0 \\
            0 & {}^0 R_n^{-1}
        \end{bmatrix} {}^0 J
    &= \begin{bmatrix}
            {}^0 R_n^T & 0 \\
            0 & {}^0 R_n^T
        \end{bmatrix} {}^0 J
\end{aligned}
\]

下面求解基座雅可比矩阵\({}^0J\)

基于D-H参数,求得各关节齐次变换矩阵如下:
\[
\begin{aligned}
    {}^0 T_1 &= \begin{bmatrix}
                    \cos\theta_1 & 0 & -\sin\theta_1 & 0 \\
                    \sin\theta_1 & 0 & \cos\theta_1 & 0 \\
                    0 & -1 & 0 & d_1 \\
                    0 & 0 & 0 & 1
                \end{bmatrix} \\
    {}^1 T_2 &= \begin{bmatrix}
                    \cos\theta_2 & -\sin\theta_2 & 0 & a_2\cos\theta_2 \\
                    \sin\theta_2 & \cos\theta_2 & 0 & a_2\sin\theta_2 \\
                    0 & 0 & 1 & 0 \\
                    0 & 0 & 0 & 1
                \end{bmatrix} \\
    {}^2 T_3 &= \begin{bmatrix}
                   \cos\theta_3 & -\sin\theta_3 & 0 & a_3\cos\theta_3 \\
                   \sin\theta_3 & \cos\theta_3 & 0 & a_3\sin\theta_3 \\
                   0 & 0 & 1 & 0 \\
                   0 & 0 & 0 & 1
               \end{bmatrix} \\
\end{aligned}
\]

进一步可计算得:
\[
\begin{aligned}
    {}^0 T_1 &= \begin{bmatrix}
                    \cos\theta_1 & 0 & -\sin\theta_1 & 0 \\
                    \sin\theta_1 & 0 & \cos\theta_1 & 0 \\
                    0 & -1 & 0 & d_1 \\
                    0 & 0 & 0 & 1
                \end{bmatrix} \\
    {}^0 T_2 &= \begin{bmatrix}
                    \cos\theta_1\cos\theta_2 & -\cos\theta_1\sin\theta_2 & -\sin\theta_1 & a_2\cos\theta_1\cos\theta_2 \\
                    \sin\theta_1\cos\theta_2 & -\sin\theta_1\sin\theta_2 & \cos\theta_1 & a_2\sin\theta_1\cos\theta_2 \\
                    -\sin\theta_2 & -\cos\theta_2 & 0 & d_1 - a_2\sin\theta_2 \\
                    0 & 0 & 0 & 1
                \end{bmatrix} \\
    {}^0 T_3 &= \begin{bmatrix}
                    \cos\theta_1\cos(\theta_2+\theta_3) & -\cos\theta_1\sin(\theta_2+\theta_3) & -\sin\theta_1 & \cos\theta_1\big(a_2\cos\theta_2 + a_3\cos(\theta_2+\theta_3)\big) \\
                    \sin\theta_1\cos(\theta_2+\theta_3) & -\sin\theta_1\sin(\theta_2+\theta_3) & \cos\theta_1 & \sin\theta_1\big(a_2\cos\theta_2 + a_3\cos(\theta_2+\theta_3)\big) \\
                    -\sin(\theta_2+\theta_3) & -\cos(\theta_2+\theta_3) & 0 & d_1 - a_2\sin\theta_2 - a_3\sin(\theta_2+\theta_3) \\
                    0 & 0 & 0 & 1
                \end{bmatrix} \\
\end{aligned}
\]

故构造基座雅可比矩阵所需各参数如下:
\[
\begin{aligned}
    {}^0 \xi_0 &= \begin{bmatrix}
                    0 \\ 0 \\ 1
                \end{bmatrix} \quad
    {}^0 \xi_1 = \begin{bmatrix}
                    -\sin\theta_1 \\ \cos\theta_1 \\ 0
                \end{bmatrix} \quad
    {}^0 \xi_2 = \begin{bmatrix}
                    -\sin\theta_1 \\ \cos\theta_1 \\ 0
                \end{bmatrix} \\
    {}^0 p_0 &= \begin{bmatrix}
                    0 \\ 0 \\ 0
                \end{bmatrix} \quad
    {}^0 p_1 = \begin{bmatrix}
                    0 \\ 0 \\ d_1
                \end{bmatrix} \quad
    {}^0 p_2 = \begin{bmatrix}
                    a_2\cos\theta_1\cos\theta_2 \\
                    a_2\sin\theta_1\cos\theta_2 \\
                    d_1 - a_2\sin\theta_2
                \end{bmatrix} \\
    {}^0 p_3 &= \begin{bmatrix}
                    \cos\theta_1\big(a_2\cos\theta_2 + a_3\cos(\theta_2+\theta_3)\big) \\
                    \sin\theta_1\big(a_2\cos\theta_2 + a_3\cos(\theta_2+\theta_3)\big) \\
                    d_1 - a_2\sin\theta_2 - a_3\sin(\theta_2+\theta_3)
                \end{bmatrix} \\
\end{aligned}
\]

构造得基座雅可比矩阵各分块如下:
\[
\begin{aligned}
    {}^0 J_{v1} &= {}^0\xi_0 \times ({}^0 p_3 - {}^0 p_0)
                = \begin{bmatrix}
                    -\sin\theta_1\big(a_2\cos\theta_2 + a_3\cos(\theta_2+\theta_3)\big) \\
                    \cos\theta_1\big(a_2\cos\theta_2 + a_3\cos(\theta_2+\theta_3)\big) \\
                    0
                \end{bmatrix} \\
    {}^0 J_{v2} &= {}^0\xi_1 \times ({}^0 p_3 - {}^0 p_1)
                = \begin{bmatrix}
                    -\cos\theta_1\big(a_2\sin\theta_2 + a_3\sin(\theta_2+\theta_3)\big) \\
                    -\sin\theta_1\big(a_2\sin\theta_2 + a_3\sin(\theta_2+\theta_3)\big) \\
                    -\big(a_2\cos\theta_2 + a_3\cos(\theta_2+\theta_3)\big)
                \end{bmatrix} \\
    {}^0 J_{v3} &= {}^0\xi_2 \times ({}^0 p_3 - {}^0 p_2)
                = \begin{bmatrix}
                    -a_3\cos\theta_1\sin(\theta_2+\theta_3) \\
                    -a_3\sin\theta_1\sin(\theta_2+\theta_3) \\
                    -a_3\cos(\theta_2+\theta_3)
                \end{bmatrix} \\
    {}^0 J_{\omega1} &= {}^0\xi_0
                = \begin{bmatrix}
                    0 \\ 0 \\ 1
                \end{bmatrix} \\
    {}^0 J_{\omega2} &= {}^0\xi_1
                = \begin{bmatrix}
                    -\sin\theta_1 \\ \cos\theta_1 \\ 0
                \end{bmatrix} \\
    {}^0 J_{\omega3} &= {}^0\xi_2
                = \begin{bmatrix}
                    -\sin\theta_1 \\ \cos\theta_1 \\ 0
                \end{bmatrix} \\
\end{aligned}
\]

为转换到末端雅可比矩阵,计算可得:
\[
\begin{aligned}
    {}^0 R_n^{\top} \; {}^0 J_{v1} &= \begin{bmatrix}
                    0 \\
                    0 \\
                    a_2\cos\theta_2 + a_3\cos(\theta_2+\theta_3)
                \end{bmatrix} \\
    {}^0 R_n^{\top} \; {}^0 J_{v2} &= \begin{bmatrix}
                    a_2\sin\theta_3 \\
                    a_2\cos\theta_3 + a_3 \\
                    0
                \end{bmatrix} \\
    {}^0 R_n^{\top} \; {}^0 J_{v3} &= \begin{bmatrix}
                    0 \\
                    a_3 \\
                    0
                \end{bmatrix} \\
    {}^0 R_n^{\top} \; {}^0 J_{\omega1} &= \begin{bmatrix}
                        -\sin(\theta_2+\theta_3) \\
                        -\cos(\theta_2+\theta_3) \\
                        0
                    \end{bmatrix} \\
    {}^0 R_n^{\top} \; {}^0 J_{\omega2} &= \begin{bmatrix}
                    0 \\
                    0 \\
                    1
                \end{bmatrix} \\
    {}^0 R_n^{\top} \; {}^0 J_{\omega3} &= \begin{bmatrix}
                    0 \\
                    0 \\
                    1
                \end{bmatrix}
\end{aligned}
\]

最终可得末端雅可比矩阵为
\[
{}^n J =
\begin{bmatrix}
    0 & a_2\sin\theta_3 & 0 \\
    0 & a_2\cos\theta_3 + a_3 & a_3 \\
    a_2\cos\theta_2 + a_3\cos(\theta_2+\theta_3) & 0 & 0 \\
    -\sin(\theta_2+\theta_3) & 0 & 0 \\
    -\cos(\theta_2+\theta_3) & 0 & 0 \\
    0 & 1 & 1
\end{bmatrix}
\]
\vspace{5em}


(2)
将\(\theta_1 = 30^\circ, \theta_2 = 50^\circ, \theta_3 = 20^\circ\)代入雅可比矩阵表达式得:
\[
\begin{aligned}
{}^n J_v &= \begin{bmatrix}
                0 & a_2\sin(20^\circ) & 0 \\
                0 & a_2\cos(20^\circ) + a_3 & a_3 \\
                a_2\cos(50^\circ) + a_3\cos(70^\circ) & 0 & 0 \\
            \end{bmatrix}
        &= \begin{bmatrix}
                0 & 0.1710 & 0 \\
                0 & 0.9698 & 0.5 \\
                0.4924 & 0 & 0 \\
            \end{bmatrix}
\end{aligned}
\]

对其进行 SVD 分解可得:
\[
{}^n J_v = U\, \Sigma\, V^{\top},\quad
\Sigma = \begin{bmatrix}
            1.1016 & 0 & 0 \\
            0 & 0.4924 & 0 \\
            0 & 0 & 0.0776
        \end{bmatrix}
\]

由此可得: \\
最小奇异值\(0.0776\) \\
条件数\(\frac{\sigma_1}{\sigma_3} = 14.19\) \\
可操作度数\(\sigma_1 \sigma_2 \sigma_3 = 0.0421\)
\vspace{5em}


(3)
将\(\theta_1 = 5^\circ, \theta_2 = 30^\circ, \theta_3 = 40^\circ\)代入雅可比矩阵表达式得:
\[
\begin{aligned}
{}^n J_v &= \begin{bmatrix}
                0 & a_2\sin(40^\circ) & 0 \\
                0 & a_2\cos(40^\circ) + a_3 & a_3 \\
                a_2\cos(30^\circ) + a_3\cos(70^\circ) & 0 & 0 \\
            \end{bmatrix}
        &= \begin{bmatrix}
                0 & 0.3214 & 0 \\
                0 & 0.8830 & 0.5 \\
                0.6040 & 0 & 0 \\
            \end{bmatrix}
\end{aligned}
\]

对其进行 SVD 分解可得:
\[
{}^n J_v = U\, \Sigma\, V^{\top},\quad
\Sigma = \begin{bmatrix}
            1.0535 & 0 & 0 \\
            0 & 0.6040 & 0 \\
            0 & 0 & 0.1525
        \end{bmatrix}
\]

由此可得: \\
最小奇异值\(0.1525\) \\
条件数\(\frac{\sigma_1}{\sigma_3} = 6.91\) \\
可操作度数\(\sigma_1 \sigma_2 \sigma_3 = 0.0971\)
\newpage



2.
(1)
基于D-H参数,求得各关节齐次变换矩阵如下:
\[
\begin{aligned}
    {}^0 T_1 &= \begin{bmatrix}
                    \cos\theta_1 & 0 & -\sin\theta_1 & 0 \\
                    \sin\theta_1 & 0 & \cos\theta_1 & 0 \\
                    0 & -1 & 0 & d_1 \\
                    0 & 0 & 0 & 1
                \end{bmatrix} \\
    {}^1 T_2 &= \begin{bmatrix}
                    \cos\theta_2 & 0 & -\sin\theta_2 & 0 \\
                    \sin\theta_2 & 0 & \cos\theta_2 & 0 \\
                    0 & -1 & 0 & 0 \\
                    0 & 0 & 0 & 1
                \end{bmatrix} \\
    {}^2 T_3 &= \begin{bmatrix}
                    \cos\theta_3 & -\sin\theta_3 & 0 & 0 \\
                    \sin\theta_3 & \cos\theta_3 & 0 & 0 \\
                    0 & 0 & 1 & d_3 \\
                    0 & 0 & 0 & 1
                \end{bmatrix} \\
\end{aligned}
\]

进一步可计算得:
\[
\begin{aligned}
    {}^0 T_1 &= \begin{bmatrix}
                    \cos\theta_1 & 0 & -\sin\theta_1 & 0 \\
                    \sin\theta_1 & 0 & \cos\theta_1 & 0 \\
                    0 & -1 & 0 & d_1 \\
                    0 & 0 & 0 & 1
                \end{bmatrix} \\
    {}^0 T_2 &= \begin{bmatrix}
                    \cos\theta_1\cos\theta_2 & \sin\theta_1 & -\cos\theta_1\sin\theta_2 & 0 \\
                    \sin\theta_1\cos\theta_2 & -\cos\theta_1 & -\sin\theta_1\sin\theta_2 & 0 \\
                    -\sin\theta_2 & 0 & -\cos\theta_2 & d_1 \\
                    0 & 0 & 0 & 1
                \end{bmatrix} \\
    {}^0 T_3 &= \begin{bmatrix}
                    c\theta_1\, c\theta_2\, c\theta_3 + s\theta_1\, s\theta_3 & -c\theta_1\, c\theta_2\, s\theta_3 + s\theta_1\, c\theta_3 & -c\theta_1\, s\theta_2 & -d_3\, c\theta_1\, s\theta_2 \\
                    s\theta_1\, c\theta_2\, c\theta_3 - c\theta_1\, s\theta_3 & -s\theta_1\, c\theta_2\, s\theta_3 - c\theta_1\, c\theta_3 & -s\theta_1\, s\theta_2 & -d_3\, s\theta_1\, s\theta_2 \\
                    -s\theta_2\, c\theta_3 & s\theta_2\, s\theta_3 & -c\theta_2 & d_1 - d_3\, c\theta_2 \\
                    0 & 0 & 0 & 1
                \end{bmatrix}
\end{aligned}
\]

故构造基座雅可比矩阵所需各参数如下:
\[
\begin{aligned}
    {}^0 \xi_0 &= \begin{bmatrix}
                    0 \\ 0 \\ 1
                \end{bmatrix} \quad
    {}^0 \xi_1 = \begin{bmatrix}
                    -\sin\theta_1 \\ \cos\theta_1 \\ 0
                \end{bmatrix} \quad
    {}^0 \xi_2 = \begin{bmatrix}
                    -\cos\theta_1 \sin\theta_2 \\ -\sin\theta_1 \sin\theta_2 \\ -\cos\theta_2
                \end{bmatrix} \\
    {}^0 p_0 &= \begin{bmatrix}
                    0 \\ 0 \\ 0
                \end{bmatrix} \quad
    {}^0 p_1 = \begin{bmatrix}
                    0 \\ 0 \\ d_1
                \end{bmatrix} \quad
    {}^0 p_2 = \begin{bmatrix}
                    0 \\ 0 \\ d_1
                \end{bmatrix} \quad
    {}^0 p_3 = \begin{bmatrix}
                    -d_3 \cos\theta_1 \sin\theta_2 \\
                    -d_3 \sin\theta_1 \sin\theta_2 \\
                    d_1 - d_3 \cos\theta_2
                \end{bmatrix} \\
\end{aligned}
\]

构造得基座雅可比矩阵各分块如下:
\[
\begin{aligned}
    {}^0 J_{v1} &= {}^0\xi_0 \times ({}^0 p_3 - {}^0 p_0)
                = \begin{bmatrix}
                    d_3 \sin\theta_1 \sin\theta_2 \\
                    -d_3 \cos\theta_1 \sin\theta_2 \\
                    0
                \end{bmatrix} \\
    {}^0 J_{v2} &= {}^0\xi_1 \times ({}^0 p_3 - {}^0 p_1)
                = \begin{bmatrix}
                    -d_3 \cos\theta_1 \cos\theta_2 \\
                    -d_3 \sin\theta_1 \cos\theta_2 \\
                    d_3 \sin\theta_2
                \end{bmatrix} \\
    {}^0 J_{v3} &= {}^0\xi_2 \times ({}^0 p_3 - {}^0 p_2)
                = \begin{bmatrix}
                    0 \\
                    0 \\
                    0
                \end{bmatrix} \\
    {}^0 J_{\omega1} &= {}^0\xi_0
                = \begin{bmatrix}
                    0 \\ 0 \\ 1
                \end{bmatrix} \\
    {}^0 J_{\omega2} &= {}^0\xi_1
                = \begin{bmatrix}
                    -\sin\theta_1 \\ \cos\theta_1 \\ 0
                \end{bmatrix} \\
    {}^0 J_{\omega3} &= {}^0\xi_2
                = \begin{bmatrix}
                    -\cos\theta_1 \sin\theta_2 \\ -\sin\theta_1 \sin\theta_2 \\ -\cos\theta_2
                \end{bmatrix} \\
\end{aligned}
\]

最终可得基座雅可比矩阵为
\[
{}^0 J =
\begin{bmatrix}
    d_3 \sin\theta_1 \sin\theta_2 & -d_3 \cos\theta_1 \cos\theta_2 & 0 \\
    -d_3 \cos\theta_1 \sin\theta_2 & -d_3 \sin\theta_1 \cos\theta_2 & 0 \\
    0 & d_3 \sin\theta_2 & 0 \\
    0 & -\sin\theta_1 & -\cos\theta_1 \sin\theta_2 \\
    0 & \cos\theta_1 & -\sin\theta_1 \sin\theta_2 \\
    1 & 0 & -\cos\theta_2
\end{bmatrix}
\]
\vspace{5em}


(2)
将\(\theta_1 = 60^\circ, \theta_2 = 100^\circ, \theta_3 = 50^\circ\)代入基座角速度雅可比分块表达式得:
\[
\begin{aligned}
{}^0 J_{\omega}
&= \begin{bmatrix}
        0 & -\sin(60^\circ) & -\cos(60^\circ)\sin(100^\circ) \\
        0 & \cos(60^\circ) & -\sin(60^\circ)\sin(100^\circ) \\
        1 & 0 & -\cos(100^\circ)
    \end{bmatrix}
 = \begin{bmatrix}
        0 & -0.8660 & -0.4924 \\
        0 & 0.5000 & -0.8529 \\
        1.0000 & 0 & 0.1736
    \end{bmatrix}
\end{aligned}
\]

对其进行 SVD 分解可得:
\[
{}^0 J_{\omega} = U\, \Sigma\, V^{\top},\quad
\Sigma = \begin{bmatrix}
            1.0834 & 0 & 0 \\
            0 & 1.0000 & 0 \\
            0 & 0 & 0.9090
        \end{bmatrix}
\]

由此可得: \\
最小奇异值\(0.9090\) \\
条件数\(\frac{\sigma_1}{\sigma_3} = 1.19\) \\
可操作度数\(\sigma_1 \sigma_2 \sigma_3 = 0.9848\)
\vspace{5em}


(3)
将\(\theta_1 = 50^\circ, \theta_2 = 178^\circ, \theta_3 = 60^\circ\)代入基座角速度雅可比分块表达式得:
\[
\begin{aligned}
{}^0 J_{\omega}
&= \begin{bmatrix}
        0 & -\sin(50^\circ) & -\cos(50^\circ)\sin(178^\circ) \\
        0 & \cos(50^\circ) & -\sin(50^\circ)\sin(178^\circ) \\
        1 & 0 & -\cos(178^\circ)
    \end{bmatrix}
 = \begin{bmatrix}
        0 & -0.7660 & -0.0224 \\
        0 & 0.6428 & -0.0267 \\
        1.0000 & 0 & 0.9994
    \end{bmatrix}
\end{aligned}
\]

对其进行 SVD 分解可得:
\[
{}^0 J_{\omega} = U\, \Sigma\, V^{\top},\quad
\Sigma = \begin{bmatrix}
            1.4140 & 0 & 0 \\
            0 & 1.0000 & 0 \\
            0 & 0 & 0.0247
        \end{bmatrix}
\]

由此可得: \\
最小奇异值\(0.0247\) \\
条件数\(\frac{\sigma_1}{\sigma_3} = 57.29\) \\
可操作度数\(\sigma_1 \sigma_2 \sigma_3 = 0.0349\)

\end{document} 