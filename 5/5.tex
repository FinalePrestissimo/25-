\documentclass[UTF8, 13pt]{ctexart}
\usepackage{amsmath}
\usepackage{graphicx}
\usepackage{physics}

% 取消段首缩进
\setlength{\parindent}{0pt}

\begin{document}

1.
(1)
由于
\[
{}^0 J = \begin{bmatrix}
            {}^0 R_n & 0 \\
            0 & {}^0 R_n
        \end{bmatrix} {}^n J
\]
其中\({}^0 R_n\)为正交矩阵,故
\[
\begin{aligned}
    {}^n J &= \begin{bmatrix}
            {}^0 R_n & 0 \\
            0 & {}^0 R_n
            \end{bmatrix}^{-1} 
            {}^0 J \\
    &= \begin{bmatrix}
            {}^0 R_n^{-1} & 0 \\
            0 & {}^0 R_n^{-1}
        \end{bmatrix} {}^0 J
    &= \begin{bmatrix}
            {}^0 R_n^T & 0 \\
            0 & {}^0 R_n^T
        \end{bmatrix} {}^0 J
\end{aligned}
\]

下面求解基座雅可比矩阵\({}^0J\)

基于D-H参数,求得各关节齐次变换矩阵如下:
\[
\begin{aligned}
    {}^0 T_1 &= \begin{bmatrix}
                    \cos\theta_1 & -\sin\theta_1\cos\alpha_1 & \sin\theta_1\sin\alpha_1 & a_1\cos\theta_1 \\
                    \sin\theta_1 & \cos\theta_1\cos\alpha_1 & -\cos\theta_1\sin\alpha_1 & a_1\sin\theta_1 \\
                    0 & \sin\alpha_1 & \cos\alpha_1 & d_1 \\
                    0 & 0 & 0 & 1
                \end{bmatrix} \\
    {}^1 T_2 &= \begin{bmatrix}
                    \cos\theta_2 & -\sin\theta_2\cos\alpha_2 & \sin\theta_2\sin\alpha_2 & a_2\cos\theta_2 \\
                    \sin\theta_2 & \cos\theta_2\cos\alpha_2 & -\cos\theta_2\sin\alpha_2 & a_2\sin\theta_2 \\
                    0 & \sin\alpha_2 & \cos\alpha_2 & d_2 \\
                    0 & 0 & 0 & 1
                \end{bmatrix} \\
    {}^2 T_3 &= \begin{bmatrix}
                   \cos\theta_3 & -\sin\theta_3\cos\alpha_3 & \sin\theta_3\sin\alpha_3 & a_3\cos\theta_3 \\
                   \sin\theta_3 & \cos\theta_3\cos\alpha_3 & -\cos\theta_3\sin\alpha_3 & a_3\sin\theta_3 \\
                   0 & \sin\alpha_3 & \cos\alpha_3 & d_3 \\
                   0 & 0 & 0 & 1
               \end{bmatrix} \\
\end{aligned}
\]

故可构造得雅可比矩阵各分块为
\[
\begin{aligned}
    {}^0 J_1 &= \begin{bmatrix}
                    \xi_1 \times ({}^0 p_3 - {}^0 P_0) \\ \xi_1
                \end{bmatrix} \\
            &= \begin{bmatrix}
                    \begin{bmatrix}
                        -\sin\theta_1 \\ \cos\theta_1 \\ 0
                    \end{bmatrix} \times
                    \begin{bmatrix}
                        0.5 \cos\theta_2 + 0.5 \cos(\theta_2 + \theta_3) \\
                        0.5 \sin\theta_2 + 0.5 \sin(\theta_2 + \theta_3) \\
                        0.3
                    \end{bmatrix} \\
                    \begin{bmatrix}
                        0 \\ 0 \\ 1
                    \end{bmatrix}
                \end{bmatrix} \\
    {}^0 J_2 &= \begin{bmatrix}
                    \xi_2 \times ({}^0 p_3 - {}^0 P_1) \\ \xi_2
                \end{bmatrix} \\
    {}^0 J_3 &= \begin{bmatrix}
                    \xi_3 \times ({}^0 p_3 - {}^0 P_2) \\ \xi_3
                \end{bmatrix} \\
\end{aligned}
\]

\end{document} 